    
\documentclass[11pt]{article}
\usepackage{times}
    \usepackage{fullpage}
    
    \title{Audio feedback for gesture recognition}
    \author{John Williamson 9804750w}

    \begin{document}
    \maketitle
    
    
     



\section{Proposal}\label{proposal}

\subsection{Motivation}\label{motivation}

Gesture recognition offers the opportunity to add controls to a myriad of sensing
devices, particularly on mobile devices with limited control inputs. However, the
resulting interfaces are hard to interpret. Adding auditory feedback to indicate
the progress and success of gesture recognition could improve the usability of
gesture recognition systems.


\subsection{Aims}\label{aims}

This project will develop a software framework for systematically exploring
audio feedback options for gesture recognition. This will be a modular visual
programming environment that allows various gesture recognisers to be configured
and their output processed and fed to audio synthesis devices. The effectiveness
of the project in improving gesture usability will be experimentally validated.

\section{Progress}\label{progress}

\begin{itemize}
    \tightlist
\item Language and GUI framework chosen: project will be implemented in Java,
using Swing for GUI development.
\item Software architecture outlined and basic class structure written.
\item  Background research conducted on gesture recognition technologies and
feedback mechanisms.
\item  Interfacing to inertial sensing unit in Java completed.
\item  Initial version of GUI developed, which allows basic signal processing to
be applied to mouse input, with interchangable blocks.
\item  Basic finite state machine gesture recogniser implemented.
\item  Initial MIDI note based output implemented and working. Limited to
pitch mapping.
\end{itemize}

\section{Problems and risks}\label{problems-and-risks}

\subsection{Problems}\label{problems}

The following issues were encountered in the project so far.
\begin{itemize}
    \tightlist
\item Inertial sensing unit had unsupported and out-of-date drivers. Some tricky
fixes had to be applied to get inputs.
\item Many different types of gesture recognition; not clear which ones to focus
on.
\item Implemented FSM recogniser is not robust.
\item Significant latency issues in rendering audio via Java with default audio
generation libraries.
\end{itemize}

\subsection{Risks}\label{risks}

\begin{itemize}
\tightlist
\item   Many different gesture recognisers to explore. \textbf{Mitigation}: will narrow
down to three possibilities by start of next semester.
\item Unclear how to evaluate success of the project. \textbf{Mitigation}: will do
background research to investigate how success of audio recognition has
been performed in the research literature.
\item Inertial sensing device seems to be unreliable. \textbf{No clear mitigation available at this stage}
\end{itemize}
    
\section{Plan}\label{plan}

\subsection{Semester 2}

\begin{itemize}
    \tightlist
    \item
      Week 1-2: develop visual programming interface. \textbf{Deliverable:}
      complete interface that allows components to be added, removed and
      rearranged.
    \item
      Week 3-5: implement three recognisers and test them with a standard
      recognition task. \textbf{Deliverable:} tested recognisers with
      initial performance metrics and integration with visual programming
      environment.
    \item
      Week 6: research on how to best evaluate performance of final system.
      \textbf{Deliverable:} detailed evaluation plan, with participant
      numbers, information sheet and analysis plan.
    \item
      Week 7-9: final implementation and improvements to audio rendering.
      \textbf{Deliverable: polished software ready, passing basic tests,
      ready for evaluation stage.}
    \item
      Week 9: evaluation experiments run. \textbf{Deliverable: quantitative
      measures of usability and qualitative measures of effectiveness for at
      least ten users.}
    \item
      Week 8-10: Write up. \textbf{Deliverable: first draft submitted to
      supervisor two weeks before final deadline.}
    \end{itemize}
    

\section{Ethics}

This project will involve tests with human users.  These will be user studies
using standard hardware, and require no personally identifiable information to be captured.
I have verified that the experiments I plan to do comply with the Ethics Checklist.

\end{document}
